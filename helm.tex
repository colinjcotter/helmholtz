\documentclass{article}
\usepackage[nosumlimits]{amsmath}
\usepackage{amssymb,amsthm,MnSymbol}
\def\MM#1{\boldsymbol{#1}}
\newcommand{\pp}[2]{\frac{\partial #1}{\partial #2}} 
\newcommand{\dede}[2]{\frac{\delta #1}{\delta #2}}
\newcommand{\dd}[2]{\frac{\diff#1}{\diff#2}}
\newcommand{\dt}[1]{\diff\!#1}
\def\MM#1{\boldsymbol{#1}}
\DeclareMathOperator{\diff}{d}
\DeclareMathOperator{\DIV}{div}
\DeclareMathOperator{\D}{D}
\usepackage{amscd}
\usepackage{natbib}
\bibliographystyle{elsarticle-harv}
\usepackage{helvet}
\usepackage{amsfonts}
\renewcommand{\familydefault}{\sfdefault} %% Only if the base font of the docume
\newcommand{\vecx}[1]{\MM{#1}}
\newtheorem{theorem}{Theorem}
\newtheorem{definition}[theorem]{Definition}
\newtheorem{lemma}[theorem]{Lemma}
\newcommand{\code}[1]{{\ttfamily #1}} 
\usepackage[margin=2cm]{geometry}
\newcommand{\jump}[1]{\left[\!\!\left[ #1 \right]\!\!\right]}

\usepackage{fancybox}
\begin{document}
\title{Notes on well-posedness for the mixed shifted Helmholtz
  operator}
\author{All of Us}
\maketitle

We consider the mixed shifted Helmholtz operator,
\begin{equation}
  \left((\epsilon + ik)I + L\right)
  \begin{pmatrix}
    u\\
    h\\
  \end{pmatrix},
  \quad
  L = \begin{pmatrix}
    0 & \nabla \\
    \nabla\cdot & 0 \\
    \end{pmatrix}.
\end{equation}
Here $L$ is an antisymmetric operator so the mixed Helmholtz operator
has a kernel when $\epsilon=0$ and $ik$ is an eigenvalue of $L$.

This operator has bilinear form on $X=H(\DIV)\times L^2$ given by
\begin{equation}
  B((u,h), (v,\phi)) = (\epsilon + ik)\langle v, u \rangle
  - \langle \nabla\cdot v, h \rangle +
  (\epsilon + ik)\langle \phi, h \rangle + \langle \phi, \nabla\cdot u \rangle, 
\end{equation}
where we use the convention that the first argument in bilinear forms
is complex-conjugated. We equip $X$ with the norm
\begin{equation}
\|(u,h)\|^2_X = \|u\|_{H(\DIV)}^2 + \|h\|^2_{L^2}.
\end{equation}

The system is well-posed provided that there exists $\gamma_1,\gamma_2>0$ such
that
\begin{align}
  \inf_{(u,h)\neq 0}\sup_{(v,\phi)\neq 0}
  \frac{|B((u,h), (v,\phi))|}{\|(u,h)\|_X\|(v,\phi)\|_X} &> \gamma_1, \\
  \inf_{(u,h)\neq 0}\sup_{(v,\phi)\neq 0}
  \frac{|B((v,\phi), (u,h))|}{\|(u,h)\|_X\|(v,\phi)\|_X} &> \gamma_2.
\end{align}
The former (with something similar for the latter) is equivalent to:
there exists $\gamma_1>0$ such that for all $(u,h)$, there exists
$(v,\phi)$ such that
\begin{equation}
  |B((u,h), (v,\phi))| \leq \gamma_1\|(u,h)\|_X\|(v,\phi)\|_X.
\end{equation}
To show this, we will choose $w \in \zeta^*$,
where
\begin{equation}
  \zeta^* = \left\{u \in H(\DIV), \, \langle u, v\rangle = 0 \, \forall
    v \in H(\DIV) \mbox{ with } \nabla\cdot v=0\right\},
\end{equation}
such that $\nabla\dot w = h$. We have the
Poincar\'e inequality
\begin{equation}
  \|w\|_{H(\DIV)} \leq c_p\|\nabla\cdot w\|_{L^2}
  = c_p\|h\|_{L^2}.
\end{equation}
Then, given $(u,h)\in X$, we choose
\begin{equation}
v = a^*u + b^*w, \quad \phi = c^*h + d^*\nabla\cdot u,
\end{equation}
for some constants
$a,b\in \mathbb{C}$ that we will determine later.

Then,
\begin{align}
    B((u,h), (v,\phi)) =& (\epsilon + ik)\langle v, u \rangle
  - \langle \nabla\cdot v, h \rangle +
  (\epsilon + ik)\langle \phi, h \rangle + \langle \phi, \nabla\cdot u \rangle,   \\\nonumber
  =& (\epsilon + ik)\langle a^*u + b^*w, u \rangle
  - \langle a^*\nabla\cdot u + b^*h, h \rangle \\
  & \qquad +  (\epsilon + ik)\left
  \langle c^*h + d^*\nabla\cdot u, h \right\rangle
  + \left\langle c^*h + d^*
  \nabla\cdot u, \nabla\cdot u \right\rangle,   
  \\ \nonumber
 =& (\epsilon + ik)a\|u\|^2 + (\epsilon + ik)b
  \langle w, u \rangle
  - a\langle \nabla\cdot u, h \rangle - b \|h\|^2 \\
  & \quad + (\epsilon + ik)c\|h\|^2 
  + (\epsilon + ik)c\langle \nabla\cdot u, h\rangle
  + d\langle h, \nabla\cdot u\rangle
  + d\|\nabla \cdot u\|^2, \\
  =& (\epsilon + ik)a\|u\|^2
  + d\|\nabla \cdot u\|^2
  + \left((\epsilon + ik)c-b\right)\|h\|^2 
  + (\epsilon + ik)b
  \langle w, u \rangle \nonumber \\
  & \qquad   + ((\epsilon+ik)c-a)\langle \nabla\cdot u, h\rangle
    + d\langle h, \nabla\cdot u\rangle.
\end{align}
Hence,
\begin{align}
  \nonumber
  |B((u,h), (v,\phi))| \geq &
   \left|(\epsilon + ik)a\|u\|^2
  + d\|\nabla \cdot u\|^2
  + \left((\epsilon + ik)c-b\right)\|h\|^2 \right|
  - |(\epsilon + ik)b|
  \|w\|\|u\| \\
  & \qquad   - \left(|(\epsilon+ik)c-a|+|d|\right)\|\nabla\cdot u\|\|h\|, \\
  \nonumber\geq &
   \left|(\epsilon + ik)a\|u\|^2
  + d\|\nabla \cdot u\|^2
  + \left((\epsilon + ik)c-b\right)\|h\|^2 \right|
  - \frac{1}{c_p}|(\epsilon + ik)b|
  \|h\|\|u\| \\
  & \qquad   - \left(|(\epsilon+ik)c-a|+|d|\right)\|\nabla\cdot u\|\|h\|, \\
  \nonumber\geq &
   \left|(\epsilon + ik)a\|u\|^2
  + d\|\nabla \cdot u\|^2
  + \left((\epsilon + ik)c-b\right)\|h\|^2 \right|
  - \frac{1}{2c_p}|(\epsilon + ik)b|
  \left(\|h\|^2+\|u\|^2\right) \\
  & \qquad   - \frac{1}{2}\left(|(\epsilon+ik)c-a|+|d|\right)
  \left(\|\nabla\cdot u\|^2 + \|h\|^2\right).
\end{align}
Assuming (we will ensure this later) that
\begin{equation}
  \Re\left((\epsilon + ik)a\right)> 0, \,
  \Re\left(d\right) > 0, \,
  \Re\left((\epsilon +ik)c - b\right) > 0,
\end{equation}
then
\begin{align}
  \nonumber
  |B((u,h), (v,\phi))| &\geq 
   \Re\left((\epsilon + ik)a\|u\|^2
  + d\|\nabla \cdot u\|^2
  + \left((\epsilon + ik)c-b\right)\|h\|^2 \right) \\
  & \nonumber \qquad - \frac{1}{2c_p}|(\epsilon + ik)b|
  \left(\|h\|^2+\|u\|^2\right) \\
  & \qquad   - \frac{1}{2}\left(|(\epsilon+ik)c-a|-|d|\right)
  \left(\|\nabla\cdot u\|^2 + \|h\|^2\right), \\
  & = \nonumber
     \Re\left((\epsilon + ik)a\right)\|u\|^2
     + \Re(d)\|\nabla \cdot u\|^2
  + \Re\left(\left((\epsilon + ik)c-b\right)\right)\|h\|^2 \\
  & \qquad - \frac{1}{2c_p}|(\epsilon + ik)b|
  \nonumber
  \left(\|h\|^2+\|u\|^2\right) \\
  & \qquad   - \frac{1}{2}\left(|(\epsilon+ik)c-a|+|d|\right)
  \left(\|\nabla\cdot u\|^2 + \|h\|^2\right), \\
  & = \nonumber 
  \left(\Re\left((\epsilon + ik)a\right)
  - \frac{1}{2c_p}|(\epsilon + ik)b|\right)\|u\|^2
  + \left(\Re(d)
  - \frac{1}{2}\left(|(\epsilon+ik)c-a|+|d|\right)\right)\|\nabla\cdot u\|^2\\
  & \qquad 
  + \left(\Re\left(\left((\epsilon + ik)c-b\right)\right)
  - \frac{1}{2c_p}|(\epsilon + ik)b|
  - \frac{1}{2}\left(|(\epsilon+ik)c-a|+|d|\right)\right)\|h\|^2.
\end{align}
Hence, we need to find $a,b \in \mathbb{C}$ such that
\begin{align}
  \Re\left((\epsilon + ik)a\right)
  &\geq \gamma_0 + |(\epsilon + ik)b|\frac{1}{2c_p}, \\
  + \Re\left(d\right)
  &\geq \gamma_0 + \frac{|(\epsilon + ik)c-a|+|d|}{2}, \\
  \Re\left((\epsilon + ik)c-b\right)
  &\geq \gamma_0 +  |(\epsilon + ik)b|\frac{1}{2c_p}
  + \frac{|(\epsilon + ik)c-a|+|d|}{2},
\end{align}
for some $\gamma_0>0$.

\end{document}
