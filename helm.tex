\documentclass{article}
\usepackage[nosumlimits]{amsmath}
\usepackage{amssymb,amsthm,MnSymbol}
\def\MM#1{\boldsymbol{#1}}
\newcommand{\pp}[2]{\frac{\partial #1}{\partial #2}} 
\newcommand{\dede}[2]{\frac{\delta #1}{\delta #2}}
\newcommand{\dd}[2]{\frac{\diff#1}{\diff#2}}
\newcommand{\dt}[1]{\diff\!#1}
\def\MM#1{\boldsymbol{#1}}
\DeclareMathOperator{\diff}{d}
\DeclareMathOperator{\DIV}{div}
\DeclareMathOperator{\D}{D}
\usepackage{amscd}
\usepackage{natbib}
\bibliographystyle{elsarticle-harv}
\usepackage{helvet}
\usepackage{amsfonts}
\renewcommand{\familydefault}{\sfdefault} %% Only if the base font of the docume
\newcommand{\vecx}[1]{\MM{#1}}
\newtheorem{theorem}{Theorem}
\newtheorem{definition}[theorem]{Definition}
\newtheorem{lemma}[theorem]{Lemma}
\newcommand{\code}[1]{{\ttfamily #1}} 
\usepackage[margin=2cm]{geometry}
\newcommand{\jump}[1]{\left[\!\!\left[ #1 \right]\!\!\right]}

\usepackage{fancybox}
\begin{document}
\title{Notes on well-posedness for the mixed shifted Helmholtz
  operator}
\author{All of Us}
\maketitle

We consider the mixed shifted Helmholtz operator,
\begin{equation}
  \left((\epsilon + ik)I + L\right)
  \begin{pmatrix}
    u\\
    h\\
  \end{pmatrix},
  \quad
  L = \begin{pmatrix}
    0 & \nabla \\
    \nabla\cdot & 0 \\
    \end{pmatrix}.
\end{equation}
Here $L$ is an antisymmetric operator so the mixed Helmholtz operator
has a kernel when $\epsilon=0$ and $ik$ is an eigenvalue of $L$.

This operator has bilinear form on $H(\DIV)\times L^2$ given by
\begin{equation}
  B(u,h; v,\phi) = (\epsilon + ik)\langle v, u \rangle
  - \langle \diff v, p \rangle +
  (\epsilon + ik)\langle \phi, h \rangle + \langle \phi, \diff u \rangle, 
\end{equation}
where we use the convention that the first argument in bilinear forms
is complex-conjugated.

The system is well-posed provided that
\begin{equation}
  \inf_{(
\end{equation}

\end{document}
